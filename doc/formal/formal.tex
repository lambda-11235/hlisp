\documentclass[12pt]{article}
\usepackage{hyperref}
\usepackage{listings}

\title{HLisp Specification}
\author{Taran Lynn}

\begin{document}
\maketitle

\begin{abstract}
  HLisp is an attempt to make a minimal lisp with a very small core. All
  computations are done through list and symbol processing, numeric support is
  not built-in. IO is also unsupported, as everything is expected to be done
  from the REPL. As such HLisp is obviously designed for ease of implementation
  and academic research.
\end{abstract}

\tableofcontents

\pagebreak



\section{Data Types}

HLisp supports four different primitive data types, which include

\begin{description}
\item[Cons Cells] Cons cells are pairs of data. They have two slots which can
  hold any other lisp datum. If \verb!a! and \verb!b! are data, then a cons
  cell is denoted as \verb!(a . b)!.

\item[Functions] Functions should fulfill several requirements.
  \begin{enumerate}
  \item Functions should eagerly evaluate their arguments.

  \item All functions should be lexically scoped.
  \end{enumerate}

\item[Macros] Macros are data just like functions, except that their arguments
  are passed unevaluated. That is, if the symbol \verb!a! is passed as an
  argument, then the macro will bind the symbol \verb!a! to the appropriate
  variable, and not bind whatever the symbol \verb!a! is bound to in the current
  environment.

\item[Nil] Nil is a datum that represents nothing, the end of a list, or boolean
  false. This paper denotes it as ().

\item[Symbols] Symbols are named identifiers. They consist of any character the
  isn't whitespace or `(', `)', `;', or `.'.
\end{description}

There are two things that should be noted about HLisp's data. First, all data is
immutable. Second, all data that isn't nil represents boolean true, and nil
represents boolean false.

In addition to these basic data types there is a compound type, which is the
list. A list is either the nil datum or a cons cell whose second slot contains a
list. A list is denoted by a group of space separated data within parenthesis.

\begin{verbatim}
list = (datum . list)
     | ()
\end{verbatim}

\section{Syntax and Semantics}

\subsection{EBNF Grammar}

\begin{verbatim}
datum = list | quote | symbol ;

data = { datum } ;

list = '(', data, ')' ;

quote = "'", datum ;

comment = ';', { nonnewline } ;

symbol = { nonspecial } ;

nonspecial = ? any character except whitespace or '(', ')', ''', and '.'. ? ;

nonnewline = ? any character that doesn't create a new line ? ;
\end{verbatim}

\subsection{Evaluation}

There are a few evaluation rules in HLisp, mainly

\begin{enumerate}
\item All quotes \verb!'a! are expanded to \verb!(quote a)!.

\item Symbols are looked up in the environment for a datum value.

\item For a list \verb!(f ...)!, if \verb!f! corresponds to the name of a
  special form, than an action specific to that special form is performed. If
  \verb!f! is a macro or function, then it is applied to the
  arguments. Otherwise, \verb!f! is evaluated and the process is repeated.

\item Function arguments are evaluated and then passed to the function.

\item Macro arguments are passed unevaluated, but the return value of the macro
  is evaluated.
\end{enumerate}

\subsubsection{More on Macro Evaluation}

Unlike in most lisps macros can be passed around as data (though special forms
can't be). If we pass a macro to a function, and that macro is invoked, then the
result will be the same as if we replaced the variable with the macro's bounded
name. For example, consider the following expansion.

\begin{verbatim}
((lambda (m a) (m (a 'b) a)) let 'a)
=> ((lambda (a) (let (a 'b) a)) 'a)
=> ((lambda (a) ((lambda (a) a) 'b)) 'a)
=> b
\end{verbatim}

This could not be done if \verb!f! was passed as a normal function. This is a
trait unique to HLisp (to the best of my knowledge).

\section{Primitive Functions}

The following are primitive functions defined in the run-time implementation.

\begin{description}
\item[(cons x y)] Creates a cons cell whose first slot is occupied by \verb!x!
  and second is occupied by \verb!y!.

\item[(car xs)] Returns the datum in the first slot of a cons cell.

\item[(cdr xs)] Returns the datum in the second slot of a cons cell.

\item[(= x y)] Returns the symbol \verb!t! if the two objects are equal,
  otherwise it returns nil.

\item[(apply f xs)] Applies the function \verb!f! to the list \verb!xs!.

\item[(atom? x)] Returns the symbol \verb!t! is \verb!x! is not a cons cell,
  otherwise nil is returned.

\item[(eval x)] Evaluates \verb!x! in the context of the current environment.
\end{description}

\section{Special Forms}

The following are special forms, which are evaluated according to special rules.

\begin{description}
\item[(if pred x y)] Evaluates \verb!pred! and does one of two things based on
  its value. If it is nil, then \verb!y! is evaluated and returned. Otherwise,
  then \verb!x! is evaluated and returned

\item[(label name x)] Binds the symbol \verb!name! to \verb!x! so that when it's
  evaluated it returns \verb!x!. Binding is restricted to local scope within
  function definition. So \verb!(lambda (x) (label y x)) 'a! will \textbf{not}
  create a global variable \verb!y!.

\item[(lambda (args...) body) or (lambda args body)] Creates a new
  function. When evaluated this function evaluates \verb!body! with the
  variable symbols rebound to the argument's values.  Function arguments are
  either a list of symbols to be bound to positional argument, or a single
  symbol bound to a list of the arguments. This form should also handle
  closures.

\item[(macro (args...) body) or (macro args body)] Works just like
  \verb!lambda!, except that it creates a macro.

\item[(quote x)] Returns its argument \verb!x! unevaluated.
\end{description}

\end{document}