\documentclass[12pt]{article}
\usepackage{hyperref}
\usepackage{listings}

\title{HLisp Specification}
\author{Taran Lynn}

\begin{document}
\maketitle

\begin{abstract}
  HLisp is an attempt to make a minimal lisp with a very small core. All
  computations are done through list and symbol processing, and numeric support
  is not built-in. IO is also unsupported, as everything is expected to be done
  from the REPL. This makes HLisp practically just the lambda calculus extended
  with lists and symbols. As such HLisp is obviously designed for ease of
  implementation and academic research.
\end{abstract}

\tableofcontents

\pagebreak



\section{Data Types}

HLisp supports three different primitive data types, which include

\begin{description}
\item[Lists] Lists are what they sound like, lists of data.

\item[Functions] Functions should fulfill several requirements.
  \begin{enumerate}
  \item Functions should lazily evaluate their arguments.

  \item All functions should be lexically scoped.
  \end{enumerate}

\item[Symbols] Symbols are named identifiers. They consist of any character that
  isn't whitespace or `(', `)', `;', or `.'.
\end{description}

All data is immutable.

\section{Syntax and Semantics}

Comments begin with a \verb!;! and extends to the end of their line.

\subsection{EBNF Grammar}

\begin{verbatim}
datum = list | quote | symbol ;

data = { datum } ;

list = '(', data, ')' ;

quote = "'", datum ;

symbol = { nonspecial } ;

nonspecial = ? any character except whitespace or '(', ')', ''', and '.'. ? ;
\end{verbatim}

\subsection{Evaluation}

There are a few evaluation rules in HLisp, mainly

\begin{enumerate}
\item All quotes \verb!'a! are expanded to \verb!(quote a)!.

\item Symbols are looked up in the environment for a datum value.

\item For a list \verb!(f ...)!, if \verb!f! corresponds to the name of a
  special form, than an action specific to that special form is performed. If
  \verb!f! is a function, then it is applied to the arguments. Otherwise,
  \verb!f! is evaluated and the process is repeated.
\end{enumerate}

\section{Primitive Functions}

The following are primitive functions defined in the run-time implementation.
These functions force evaluation of their arguments. Note that true and false
are defined as \verb!(lambda (x y) x)! and \verb!(lambda (x y) y)!,
respectively.

\begin{description}
\item[(cons x xs)] Appends an element \verb!x! to the beginning of a list
  \verb!xs!. This is an error if \verb!xs! is not a list.

\item[(car xs)] Returns the datum at the beginning of the list \verb!xs!. This
  is an error if \verb!xs! is not a list.

\item[(cdr xs)] Returns all but the first datum at the beginning of the list
  \verb!xs!. This is an error if \verb!xs! is not a list.

\item[(= x y)] Returns true if the two objects are equal, otherwise it returns
  false. Functions are never equal to any other value.

\item[(list? x)] Returns true if \verb!x! is a list, false otherwise.
\end{description}

\section{Special Forms}

The following are special forms, which are evaluated according to special rules.

\begin{description}
\item[(label name x)] Binds the symbol \verb!name! to \verb!x! so that when it's
  evaluated it returns \verb!x!. Binding is restricted to local scope within
  function definition. So \verb!(lambda (x) (label y x))! will \textbf{not}
  create a global variable \verb!y!.

\item[lambda] Forms for lambda include
  \begin{itemize}
    \item \verb!(lambda name args body)!,
    \item \verb!(lambda args body)!,
    \item \verb!(lambda name (args...) body)!,
    \item and \verb!(lambda (args...) body)!
  \end{itemize}

  It creates a new function. When evaluated this function evaluates \verb!body!
  with the variable symbols rebound to the argument's values. Function arguments
  are either a list of symbols to be bound to positional argument, or a single
  symbol bound to a list of the arguments. This form should also handles
  closures. An optional name parameter enables recursion.

\item[(quote x)] Returns its argument \verb!x! unevaluated.

\item[(apply f xs)] Applies a function to the elements of a list.
\end{description}

\end{document}
